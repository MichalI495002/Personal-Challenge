\documentclass{article}

\usepackage{geometry}
\geometry{a4paper, margin=1in}

\usepackage{titlesec}
\titleformat{\section}{\large\bfseries}{\thesection.}{1em}{}
\titleformat{\subsection}{\bfseries}{\thesubsection.}{1em}{}

\usepackage{fancyhdr}
\usepackage{graphicx}
\usepackage{booktabs}
\usepackage[hidelinks]{hyperref}
\usepackage{xcolor}
\usepackage{enumitem}

\usepackage{tikz}
\usetikzlibrary{shapes,arrows,positioning}

\hypersetup{
    colorlinks=true,
    linkcolor=black,
    filecolor=magenta,      
    urlcolor=cyan,
    pdftitle={Overleaf Example},
    pdfpagemode=FullScreen,
}

\pagestyle{fancy}
\fancyhf{}
\rhead{\textbf{}}
\lhead{\today}
\rfoot{Page \thepage}

\begin{document}

\title{Demonstration Phase}
\date{\today}
\maketitle

\newpage
\tableofcontents
\newpage

\section{Introduction}
This document aims to introduce a cross-platform application that demonstrates the AI model to the stakeholder. The application, leveraging machine learning for sales forecasting, features a comprehensive full-stack design. The front-end is crafted using HTML, CSS, and JavaScript, while the back-end is built with Python, utilizing the Eel framework. The stakeholder oversees the sales and purchasing departments across Europe and South China. Given their role, gaining insights into sales is a pivotal factor in strategizing for each quarter of the year. The solution presented is designed to aid in making key strategic decisions.
\smallbreak
\textit{Demonstration date: 15.12.2023}
\section{Demonstration}
\subsection{Location}
The demonstration was conducted in an online meeting due to the geographical location of client
\subsection{Description}
During the session, we began with an introduction to the application, which is tailored for forecasting sales of protective gear in the Polish market. This introduction highlighted the application's main characteristics and the stakeholder's preferences. We presented a notebook detailing the workings of the model, including how it generates predictions and the rationale behind selecting specific models and their features for this particular scenario. Additionally, we addressed the limitations of the model, which stem from the unique requirements of the application and the nature of the data, especially considering the significant impact of global events on the data.


\subsection{Walkthrough}
The client effectively operated the application on their device, gaining a genuine understanding of its functionality. For testing purposes, mock data was supplied, but the client also employed their own data to replicate real-life situations. Testing of the application, including its model and predictions, faced certain constraints due to the project's nature. The challenge lay in the substantial quantity of data required by the model, which is difficult to enter manually. Nevertheless, the client quickly became familiar with the application's interface and was able to demonstrate its results effectively.

\subsection{Media}
There was provided video of usage of application for documetion: \smallbreak
\href{ https://youtu.be/mwCkFTVhNcA}{Presetaion Video}

\section{Feedback}

\subsection{Questions and Answers}
\begin{description}[style=nextline]
    \item[Does application is cross-platform?]
    Answer: Yes, application can be build to executable file for systems, macOS, Windows and Linux
    \item[Does application is easily convertible to web application?] Answer: Yes, application is created on python eel framework which is using web based system
    \item[Does application can use different kind of models?] Answer: It requires significant amount of work to apply different models and to this specific use case different models are based on seasonality 
    \item[Does user experience can be improved?] Answer: Yes, current state of application is first prototype of creating not code base way of interacting with model
    \end{description}
\subsection{Stakeholder Feedback}
The stakeholder expressed satisfaction with the fulfillment of their requirements for a web-based cross-platform application. They were particularly pleased with the inclusion of a graphical user interface that effectively displays the status of the entered data. Additionally, the interface clearly differentiates predictions from the provided data and offers a visual representation of the model's accuracy. While the client was somewhat disappointed with the model's accuracy, they were previously informed about the challenges and irregular nature of the data, which affected the precision of the predictions. Looking ahead, the client anticipates the integration of more sophisticated models, such as SARIMA or Prophet, to enhance accuracy and incorporate seasonality in future developments.


\section{Reflection}
The project encountered numerous obstacles, particularly in data preparation and determining which data was applicable and how to use it. The data's unique nature posed a significant challenge, making it difficult for the model to achieve even basic accuracy levels. This was due to factors such as the lack of seasonal patterns, the impact of the pandemic, economic crises, and political changes. The absence of seasonality rendered many models nearly useless or less accurate than a simple median. However, stakeholder were pleasantly surprised by the application, analysis, and sourcing of data. I was aware of these challenges at the project's outset, but I underestimated their extent. I wouldn't readily undertake such a challenge again due to the frustrating nature of work and the complexities involved in creating machine learning-based forecasting. I would prefer working on classification problems. Nevertheless, this challenge taught me numerous creative problem-solving techniques and about manipulating data and models to achieve better outcomes.



\end{document}