\documentclass[12pt]{article}
\usepackage{geometry}
\geometry{a4paper}
\usepackage{graphicx}
\usepackage{amsmath}
\usepackage{amsfonts}
\usepackage{amssymb}
\usepackage{booktabs}
\usepackage{enumitem}
\usepackage{hyperref}
\usepackage{titlesec}
\hypersetup{
    colorlinks=true,
    linkcolor=black,      % color of internal links
    citecolor=blue,      % color of links to bibliography
    filecolor=magenta,   % color of file links
    urlcolor=cyan,       % color of external links
    linkbordercolor={1 1 1}, % set to white
    citebordercolor={1 1 1}, % set to white
    urlbordercolor={1 1 1}   % set to white
}

\title{\textbf{Predictive Analysis of Antibiotic Resistance in Bacterial Strains via DNA Sequence Motifs}}
\author{Michał Raczkowski \\
        Fontys ICT - AI-core-AI4-RB02}
\date{\today}

\begin{document}

\begin{titlepage}
    \centering
    \vspace*{\fill}
    {\LARGE \textbf{Data Requirmetns}}
    \bigbreak
    \bigbreak

    {\LARGE {Predictive Analysis of Antibiotic Resistance in Bacterial Strains via DNA Sequence Motifs}}
    \vspace{2cm}\\
    \large{Author: Michał Raczkowski}\\
    {Student number: 4465024 }\\
    {Fontys ICT - 
    AI-core-AI4-RB02
    }

    \vspace{2cm}
    \today
    \vspace*{\fill}
\end{titlepage}

\tableofcontents
\newpage

\section{Genomic Data}
\begin{description}
    \item[Sequence Data:] \
    \begin{itemize}
        \item \textbf{Type:} DNA sequences of bacterial strains.
        \item \textbf{Format:} FASTA. Each entry should have a unique identifier and the corresponding DNA sequence.
        \item \textbf{Details:} Complete genomes or specific genes associated with antibiotic resistance (e.g., \textit{bla} genes for \textit{E. coli}).
    \end{itemize} 
    \item[Metadata:] \
    \begin{itemize}
        \item \textbf{Type:} Supplementary information for each bacterial strain.
        \item \textbf{Details:} Bacterial strain identifier, source, collection date, location, clinical outcomes, resistance phenotypes, etc.
    \end{itemize}
\end{description}

\section*{Antibiotic Resistance Phenotype Data}
\begin{description}[leftmargin=1cm]
    \item[Resistance Profile:] \
    \begin{itemize}
        \item \textbf{Type:} Results from antibiotic susceptibility tests.
        \item \textbf{Format:} Tabular data linking strain identifiers to susceptibility results.
        \item \textbf{Details:} Includes MIC values, interpretation (Resistant, Intermediate, Susceptible), antibiotic type, etc.
    \end{itemize}
\end{description}

\section{Auxiliary Data}
\begin{description}
    \item[Reference Data:] \
    \begin{itemize}
        \item \textbf{Type:} Known antibiotic resistance genes and variants.
        \item \textbf{Source:} Databases like ResFinder, CARD, etc.
        \item \textbf{Format:} FASTA or similar with sequences of known resistance genes.
        \item \textbf{Details:} Aid in identifying genes or motifs in the genome data associated with resistance.
    \end{itemize} 
    \item[Control Sequences:]\
    \begin{itemize}
        \item \textbf{Type:} DNA sequences of strains known to be susceptible.
        \item \textbf{Purpose:} For building a balanced dataset.
    \end{itemize}
\end{description}

\section{Data Quality Requirements}
\begin{itemize}
    \item \textbf{Accuracy:} Accurate sequences without errors.
    \item \textbf{Completeness:} Full genomes or comprehensive gene sequences.
    \item \textbf{Consistency:} Matching metadata and phenotype data.
    \item \textbf{Resolution:} High-resolution genomic data, preferably from WGS.
\end{itemize}

\section{Data Storage and Management}
\begin{itemize}
    \item \textbf{Database System:} A relational database system like MySQL or PostgreSQL.
    \item \textbf{Backup:} Regular data backups.
\end{itemize}

\section{Possible Data Sources}
\begin{description}
    \item[NCBI GenBank:] A comprehensive public database of nucleotide sequences and supporting bibliographic and biological annotation.
    \item[ResFinder:] A database hosted by the Center for Genomic Epidemiology, focusing specifically on antibiotic resistance genes in bacteria.
    \item[CARD (Comprehensive Antibiotic Resistance Database):] A rigorously curated collection of known resistance genes.
    \item[MicrobesOnline:] A platform offering integrated tools for visualizing and analyzing microbial genomes and their associated functional information.
    \item[ENA (European Nucleotide Archive):] A globally comprehensive data resource for nucleotide sequence, spanning raw data, alignments, and assembled/annotated sequences.
    \item[PATRIC (Pathosystems Resource Integration Center):] A bioinformatics resource center that provides comprehensive bacterial infectious disease data.
\end{description}


\end{document}
